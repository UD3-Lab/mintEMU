% Options for packages loaded elsewhere
\PassOptionsToPackage{unicode}{hyperref}
\PassOptionsToPackage{hyphens}{url}
\PassOptionsToPackage{dvipsnames,svgnames,x11names}{xcolor}
%
\documentclass[
  letterpaper,
  DIV=11,
  numbers=noendperiod]{scrartcl}

\usepackage{amsmath,amssymb}
\usepackage{lmodern}
\usepackage{iftex}
\ifPDFTeX
  \usepackage[T1]{fontenc}
  \usepackage[utf8]{inputenc}
  \usepackage{textcomp} % provide euro and other symbols
\else % if luatex or xetex
  \usepackage{unicode-math}
  \defaultfontfeatures{Scale=MatchLowercase}
  \defaultfontfeatures[\rmfamily]{Ligatures=TeX,Scale=1}
\fi
% Use upquote if available, for straight quotes in verbatim environments
\IfFileExists{upquote.sty}{\usepackage{upquote}}{}
\IfFileExists{microtype.sty}{% use microtype if available
  \usepackage[]{microtype}
  \UseMicrotypeSet[protrusion]{basicmath} % disable protrusion for tt fonts
}{}
\makeatletter
\@ifundefined{KOMAClassName}{% if non-KOMA class
  \IfFileExists{parskip.sty}{%
    \usepackage{parskip}
  }{% else
    \setlength{\parindent}{0pt}
    \setlength{\parskip}{6pt plus 2pt minus 1pt}}
}{% if KOMA class
  \KOMAoptions{parskip=half}}
\makeatother
\usepackage{xcolor}
\setlength{\emergencystretch}{3em} % prevent overfull lines
\setcounter{secnumdepth}{-\maxdimen} % remove section numbering
% Make \paragraph and \subparagraph free-standing
\ifx\paragraph\undefined\else
  \let\oldparagraph\paragraph
  \renewcommand{\paragraph}[1]{\oldparagraph{#1}\mbox{}}
\fi
\ifx\subparagraph\undefined\else
  \let\oldsubparagraph\subparagraph
  \renewcommand{\subparagraph}[1]{\oldsubparagraph{#1}\mbox{}}
\fi


\providecommand{\tightlist}{%
  \setlength{\itemsep}{0pt}\setlength{\parskip}{0pt}}\usepackage{longtable,booktabs,array}
\usepackage{calc} % for calculating minipage widths
% Correct order of tables after \paragraph or \subparagraph
\usepackage{etoolbox}
\makeatletter
\patchcmd\longtable{\par}{\if@noskipsec\mbox{}\fi\par}{}{}
\makeatother
% Allow footnotes in longtable head/foot
\IfFileExists{footnotehyper.sty}{\usepackage{footnotehyper}}{\usepackage{footnote}}
\makesavenoteenv{longtable}
\usepackage{graphicx}
\makeatletter
\def\maxwidth{\ifdim\Gin@nat@width>\linewidth\linewidth\else\Gin@nat@width\fi}
\def\maxheight{\ifdim\Gin@nat@height>\textheight\textheight\else\Gin@nat@height\fi}
\makeatother
% Scale images if necessary, so that they will not overflow the page
% margins by default, and it is still possible to overwrite the defaults
% using explicit options in \includegraphics[width, height, ...]{}
\setkeys{Gin}{width=\maxwidth,height=\maxheight,keepaspectratio}
% Set default figure placement to htbp
\makeatletter
\def\fps@figure{htbp}
\makeatother

\KOMAoption{captions}{tableheading}
\makeatletter
\makeatother
\makeatletter
\makeatother
\makeatletter
\@ifpackageloaded{caption}{}{\usepackage{caption}}
\AtBeginDocument{%
\ifdefined\contentsname
  \renewcommand*\contentsname{Table of contents}
\else
  \newcommand\contentsname{Table of contents}
\fi
\ifdefined\listfigurename
  \renewcommand*\listfigurename{List of Figures}
\else
  \newcommand\listfigurename{List of Figures}
\fi
\ifdefined\listtablename
  \renewcommand*\listtablename{List of Tables}
\else
  \newcommand\listtablename{List of Tables}
\fi
\ifdefined\figurename
  \renewcommand*\figurename{Figure}
\else
  \newcommand\figurename{Figure}
\fi
\ifdefined\tablename
  \renewcommand*\tablename{Table}
\else
  \newcommand\tablename{Table}
\fi
}
\@ifpackageloaded{float}{}{\usepackage{float}}
\floatstyle{ruled}
\@ifundefined{c@chapter}{\newfloat{codelisting}{h}{lop}}{\newfloat{codelisting}{h}{lop}[chapter]}
\floatname{codelisting}{Listing}
\newcommand*\listoflistings{\listof{codelisting}{List of Listings}}
\makeatother
\makeatletter
\@ifpackageloaded{caption}{}{\usepackage{caption}}
\@ifpackageloaded{subcaption}{}{\usepackage{subcaption}}
\makeatother
\makeatletter
\@ifpackageloaded{tcolorbox}{}{\usepackage[many]{tcolorbox}}
\makeatother
\makeatletter
\@ifundefined{shadecolor}{\definecolor{shadecolor}{rgb}{.97, .97, .97}}
\makeatother
\makeatletter
\makeatother
\ifLuaTeX
  \usepackage{selnolig}  % disable illegal ligatures
\fi
\IfFileExists{bookmark.sty}{\usepackage{bookmark}}{\usepackage{hyperref}}
\IfFileExists{xurl.sty}{\usepackage{xurl}}{} % add URL line breaks if available
\urlstyle{same} % disable monospaced font for URLs
\hypersetup{
  pdftitle={mintEMU Software Management Plan},
  pdfauthor={Claudiu Forgaci},
  colorlinks=true,
  linkcolor={blue},
  filecolor={Maroon},
  citecolor={Blue},
  urlcolor={Blue},
  pdfcreator={LaTeX via pandoc}}

\title{mintEMU Software Management Plan}
\author{Claudiu Forgaci}
\date{}

\begin{document}
\maketitle
\ifdefined\Shaded\renewenvironment{Shaded}{\begin{tcolorbox}[frame hidden, borderline west={3pt}{0pt}{shadecolor}, enhanced, boxrule=0pt, sharp corners, interior hidden, breakable]}{\end{tcolorbox}}\fi

This Software Management Plan uses the guidelines for medium management
level as described in the Practical guide to Software Management Plans
(\textbf{martinez-ortiz2022?}).

\textbf{Please provide a brief description of your software, stating its
purpose and intended audience.}

\emph{mintEMU} is a research compendium created as an R package with
\texttt{rrtools} in preparation of the scientific paper \emph{The Legacy
of the European Post-Master in Urbanism at TU Delft: A Text Mining
Approach}. The compendum contains the data, code for data processing and
analysis, as well as the paper presenting the results of the analysis.
The intended audience is the scientific community interested in the
research results, methods employed or data generated in the project.

The project was prompted by the closure of the European post-Master of
Urbanism (EMU) of the Department of Urbanism at the Faculty of
Architecture and the Built Environment, TU Delft. In order to describe
the legacy of the EMU program, including the distinctive features of its
didactic approach, this research aims to reveal the main topics taught
in it and how those topics had evolved through the years of the program.
To that end, we employ a text mining approach in which we analyse its
output: 90+ theses each with 100+ pages produced over the years for the
duration of the program between 2007-2021.

The findings will be presented in a journal article, accompanied by a
fully reproducible research compendium comprising an open dataset and
corresponding software, as well as a dashboard to enable users to
interact with the data and explore relations and patterns in the data
beyond the findings presented in the paper. The project is carried out
with in-kind support from the TU Delft Digital Competence Center.

\textbf{How will you manage versioning of your software?}

The software will be versioned with Git and hosted on GitHub to allow
for collaborative development. Although the research compendium is not a
software that will be versioned after publication, semantic versioning
will be used in the process leading up to publication, as follows:
development version 0.0.0.9000 in the research stage; patch version
0.0.1 with the publication of the preprint; minor version 0.1.0 and
subsequent increments of minor version at the time of submission for
review, and responses to reviews, respectively; and 1.0.0 for the
published paper.

\textbf{How will your software be documented for users? Please provide a
link to the documentation if available. How will you document your
software's contribution guidelines and governance structure?}

The research compendium will be documented as an R package, that is, all
functions used in data processing and analysis will have associated
documentation available from the help interface of R. Once loaded as a
package, all documentation is available to the user. The analysis will
be fully documented in a Quarto computational notebook using R Markdown
with the knitr engine.

\textbf{What licence will you give your software? How will you check
that it respects the licences of libraries and dependencies it uses?}

The software will use an MIT license.

\textbf{How will the installation requirements of your software be
documented? Please provide a link to the installation documentation if
available.}

Installation requirements are described in the
\href{https://github.com/UD3-Lab/mintEMU\#how-to-run-in-your-browser-or-download-and-run-locally}{README
file} of the repository on GitHub.

\textbf{How will users of your software be able to cite your software?
Please provide a link to your software citation file (CFF) if
available.}

Citation guidelines are described in the
\href{https://github.com/UD3-Lab/mintEMU\#how-to-cite}{README file} of
the repository on GitHub.

\textbf{How will your software be documented for future developers?}

All the functions are well documented and the following R package
structure so that they can be reused in other context. The repository
README file has also a
\href{https://github.com/UD3-Lab/mintEMU\#contributions}{Contributions}
section, where
\href{https://github.com/UD3-Lab/mintEMU/blob/main/CONTRIBUTING.md}{contributor
guidelines} and
\href{https://github.com/UD3-Lab/mintEMU/blob/main/CONDUCT.md}{Contributor
code of Conduct} can be easily accessed.

\textbf{How will your software be tested? Please provide a link to the
(automated) testing results.}

TBD

\textbf{Do you follow specific software quality guidelines? If yes,
which ones?}

FAIR4RS

\textbf{How will your software be packaged and distributed? Please
provide a link to available packaging information (e.g.~entry in a
packaging registry, if available).}

The software will be made available as an R package on GitHub, available
for direct installation from RStudio. As the package is s research
compendium, it will not be made available in a packaging registry.

The snapshot of the GitHub repository (e.g.~release associated with the
publication of the paper) will be published on Zenodo.

\textbf{How do you plan to procure long term maintenance of your
software?} The long-term maintenance of the software is not foreseen in
the project.The software is making use of the package management
software, \texttt{renv}, which help user retrieve the required version
of dependencies needed to use the functionalities of the software.



\end{document}
